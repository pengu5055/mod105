\documentclass[a4paper]{article}
\usepackage[utf8]{inputenc}
\usepackage[slovene]{babel}
\usepackage{graphicx}
\usepackage{hyperref}
\usepackage[nottoc]{tocbibind}
\usepackage{caption}
\usepackage{subcaption}
\usepackage{amsmath}
\usepackage{ dsfont }
\usepackage{siunitx}
\usepackage{multimedia}
\usepackage[table,xcdraw]{xcolor}
\usepackage{float}
\setlength\parindent{0pt}

\newcommand{\ddd}{\mathrm{d}}
\newcommand\myworries[1]{\textcolor{red}{#1}}
\newcommand{\Dd}[3][{}]{\frac{\ddd^{#1} #2}{\ddd #3^{#1}}}

\begin{document}
\begin{titlepage}
    \begin{center}
        \includegraphics[]{logo.png}
        \vspace*{3cm}
        
        \Huge
        \textbf{Modeli kemijskih reakcij}
        
        \vspace{0.5cm}
        \large
        5. naloga pri Modelski Analizi 1

        \vspace{4.5cm}
        
        \textbf{Avtor:} Marko Urbanč (28232019)\ \\
        \textbf{Predavatelj:} prof. dr. Simon Širca\ \\
        \textbf{Asistent:} doc. dr. Miha Mihovilovič\ \\
        
        \vspace{2.8cm}
        
        \large
        8.11.2023
    \end{center}
\end{titlepage}
\tableofcontents
\newpage
\section{Uvod}
Čeprav smo fiziki in smo na študiju fizike, ne moremo zanikati, da je tudi kemija pomembna veda. A izkaže se, da 
ni vedno najbolj ugodno narediti prav vsak eksperiment v živo. Zato se je razvil računalniški pristop,
ki nam omogoča, da lahko simuliramo različne kemijske reakcije. Pravzaprav je pristop praktično analogen
prejšnji nalogi o populacijskih modelih. No vsaj ko naredimo ustrezne približke. V resnici, če bi hoteli brez kompromisa
modelirati kemijske procese, bi nas to drago stalo. To mislim dobesedno, ker če bi reševali t.i. \textbf{Chemical master
equation} bi dobili sistem ODE s toliko komponent, kolikor je možnih stanj sistema. To pa je, če želimo simulirati nek
makroskopski proces, zelo veliko. Zato se bomo v tej nalogi omejili na reševanje \textbf{Reaction rate} enačb, ki so
približek prej omenjenih enačb. V tej nalogi bomo spoznali modeliranje kemijskih procesov na podlagi treh primerov.\\

\subsection{Binarna reakcija}
Za prvi primer si poglejmo primer binarne reakcije. Kemijske procese lahko opišemo kot:

\begin{gather}
    A + A \overset{p}{\underset{q}{\rightleftharpoons}} A + A^*\>,\\
    A^* \overset{r}{\rightarrow} B + C\>.
\end{gather}

Rate enačbe v brezdimenzijski obliki za proces dobimo tako, da vse koncentracije normiramo
z začetno koncentracijo snovi $A$ $[A](0)$. S tem bomo lahko uvedli še nove parametre. Takole:

\begin{gather}
    \dot{a} = \frac{1}{2} kaa^* - a^2 + \frac{1}{2}a^2\>,\\
    \dot{a^*} = \frac{1}{2}a^2 - \frac{1}{2} kaa^* - ska^*\>,\\
    \dot{b} = \frac{1}{2} ska^*\>,\\
    \dot{c} = \frac{1}{2} ska^*\>.
\end{gather}

kjer je $k = \frac{p}{q}$ in $s = \frac{r}{q[A](0)}$. Odvod pa je zdaj po $d\tau = p[A](0)\>dt$. Lahko
predpostavimo, da v ravnovesnem stanju velja $\dot{a^*} = 0$, kar nam omogoči, da izrazimo $a^*$ in s tem 
dobimo nov set enačb:

\begin{gather}
    \dot{a} = \frac{a^3}{4k(s + a/2)} - a^2\>,\\
    \dot{a^*} = 0\>,\\
    \dot{b} = \frac{sa^2}{4(s + a/2)}\>,\\
    \dot{c} = \frac{sa^2}{4(s + a/2)}\>.
\end{gather}

\subsection{Sinteza vodikovega bromida}
Za drugi primer pogledamo primer sinteze vodikovega bromida. Sinteza je sestavljena iz več stopnj. Te lahko
zapišemo kot:

\begin{gather}
    \mathrm{Br}_2 \overset{p}{\underset{q}{\rightleftharpoons}} 2\mathrm{Br}\>,\\
    \mathrm{Br} + \mathrm{H}_2 \overset{r}{\underset{s}{\rightleftharpoons}} \mathrm{HBr} + \mathrm{H}\>,\\
    \mathrm{H} + \mathrm{Br}_2 \overset{t}{\rightarrow} \mathrm{HBr} + \mathrm{Br}\>.
\end{gather}

To lahko z sistemom ODE opišemo kot prej takole:

\begin{gather}
    \dot{u} = sxy - ruz\>,\\
    \dot{v} = qz^2 - pv - \frac{1}{2}tvy\>,\\
    \dot{x} = \frac{1}{2}tvy - \frac{1}{2}sxy + \frac{1}{2}ruz\>,\\
    \dot{y} = \frac{1}{2}ruz - \frac{1}{2}sxy + \frac{1}{2}tvy\>,\\
    \dot{z} = \frac{1}{2}tvy + pv - qz^2\>,
\end{gather}

kjer je $u = [\mathrm{H}_2]$, $v = [\mathrm{Br}_2]$, $x = [\mathrm{HBr}]$, $y = [\mathrm{H}]$ in $z = [\mathrm{Br}]$.
Tudi ta sistem lahko zapišemo v približku stacionarnega stanja, kjer velja $\dot{y} = 0$ in dobimo:

\begin{gather}
    \dot{u} = -\frac{1}{2}tvy\>,\\
    \dot{v} = qz^2 - pv - \frac{1}{2}tvy\>,\\
    \dot{x} = tvy\>,\\
    \dot{y} = 0\>,\\
    \dot{z} = \frac{1}{2}tvy + pv - qz^2\>.
\end{gather}

Na predavanjih je profesor omenil še možnost stacionarnega stanja za $\dot{z} = 0$, ampak meni ni uspelo
uspešno pognati reakcije v tem primeru.\\

\subsection{Kemijska ura}
Za konec pa še bolj zabaven primer, kemijska ura. Princip je, da imamo dve reakciji, ki se odvijata vzporedno.
Prva reakcija je hitra, druga pa počasna. Ko se prva reakcija konča, se začne druga in s tem se spremeni barva
raztopine. Pravzaprav imamo dve reakciji za vsako od barv, torej dve hitri in dve počasni. To lahko zapišemo kot:

\begin{gather}
    \mathrm{S}\mathrm{O}_8^{2-} + \mathrm{I}^- \overset{p_\mathrm{slow}}{\longrightarrow} \mathrm{I}\mathrm{S}_2\mathrm{O}_8^{3-}\>,\\
    2\mathrm{S}_2\mathrm{O}_3^{2-} + \mathrm{I}_2 \overset{p_\mathrm{fast}}{\longrightarrow} \mathrm{S}_4\mathrm{O}_6^{2-} + 2\mathrm{I}^-\>,\\
    \mathrm{S}_{2}\mathrm{O}_3^{2-} + \mathrm{I_2} \overset{q_\mathrm{slow}}{\longrightarrow} \mathrm{I}\mathrm{S}_2\mathrm{O}_3^{-} + \mathrm{I}^-\>,\\
    \mathrm{I}\mathrm{S}_2\mathrm{O}_3^{-} + \mathrm{S}_2\mathrm{O}_3^{2-} \overset{q_\mathrm{fast}}{\longrightarrow} \mathrm{S}_4\mathrm{O}_6^{2-} + \mathrm{I}^-\>.
\end{gather}

To sem modeliral kar po isti metodi kot prej. Označil sem $u = [\mathrm{S}_2\mathrm{O}_8^{2-}]$,
$v = [\mathrm{I}\mathrm{S}_2\mathrm{O}_8^{3'}]$, $w = [\mathrm{S}\mathrm{O}_4^{2-}]$,
$x = [\mathrm{S}_2\mathrm{O}_3^{2-}]$, $y = [\mathrm{I}\mathrm{S}_2\mathrm{O}_3^{-}]$, $z = [\mathrm{S}_4\mathrm{O}_6^{2-}]$
in $m = [\mathrm{I}^-]$ ter $n = [\mathrm{I}_2]$. Dobimo 8D sistem:

\begin{gather}
    \dot{m} = - p_\mathrm{slow}mu - p_\mathrm{fast}mv + q_\mathrm{slow}xn + q_\mathrm{fast}xy\>,\\
    \dot{n} = p_\mathrm{fast}mv - q_\mathrm{slow}xn\>,\\
    \dot{u} = - p_\mathrm{slow}mu\>,\\
    \dot{v} = p_\mathrm{slow}mu - p_\mathrm{fast}mv\>,\\
    \dot{w} = p_\mathrm{fast}mv\>,\\
    \dot{x} = - q_\mathrm{slow}nx - q_\mathrm{fast}xy\>,\\
    \dot{y} = q_\mathrm{slow}nx - q_\mathrm{fast}xy\>,\\
    \dot{z} = q_\mathrm{fast}xy\>.
\end{gather}


\section{Naloga}
Naloga nas prosi, da za binarne reakcije integriramo sistem eksaktno in
v aproksikmaciji stacionarnega stanja za različne vrednosti $s$ pri čemer je 
$k=1000$. Za sintezo vodikovega bromida pa naj v približku stacionarnega stanja
določimo empirični konstanti $k$ in $m$ izrazu:

\begin{equation}
    \dot{[\mathrm{HBr}]} = \frac{k[\mathrm{H}_2][\mathrm{Br}_2^{1/2}]}{m + [\mathrm{HBr}]/[\mathrm{Br}_2]}
\end{equation}

Skicirajmo naj poteke za hitrost sinteze vodikovega bromida v odvisnosti od začetne konentracije 
$\mathrm{H}_2$ in $\mathrm{Br}_2$. Kaj se zgodi če primešamo mnogo $\mathrm{HBr}$?\\

Na koncu pa še narišimo potek koncentracij $m$, $n$ za kemijsko uro, kjer spreminjamo razmerje
hitrosti glavnih reakcij.\\

\section{Opis reševanja}
Za to nalogo se mi zdi, da sem bil nekoliko uninspired. Imel sem velike želje, ampak je žal moja 
spužva za razmišljanje imela druge načrte ter mi je začela delat probleme. Zato sem ostal še kar 
pri železni srajci in naredil isto kot pri prejšnji nalogi. \\

Za vsak primer sem naredil svoj razred, ki ima metode za integracijo in vsebuje enačbe modela. Takole
se lahko instancira razred za različne parametre zelo hitro in z enim klicem dobiš rešitev. V ta namen sem
uporabljal standardni nabor knjižnic za numerično računanje v Pythonu, t.j. \texttt{numpy} in \texttt{scipy} in
seveda \texttt{matplotlib} za risanje. Za integrator, kar je verjetno še najbolj zanimivo, sem uporabljal
\texttt{scipy.integrate.solve\_ivp} z metodo \texttt{RK45} in \texttt{LSODA}. \\

\section{Rezultati}
\subsection{Binarna reakcija}
Verjetno je najbolje, da kar začnemo z rezultati. Najprej bi rad samo pokazal, kakšna je rešitev za
binarno reakcijo. To je prikazano na sliki \ref{fig:binarna} za dva primera. Za prvi primer sem že vzel
parametre iz naloge, v drugem pa sem jih spremenil, tako da se vidi tudi nekaj dogajanja z $a^*$, ki sicer
v prvem primeru ostane konstantno na $0$.\\

\begin{figure}[H]
    \centering
    \makebox[\textwidth][c]{%
    \includegraphics[width=1.2\textwidth]{../BinaryReactions/Images/basic-solution.png}
    }
    \caption{Rešitev za binarno reakcijo.}
    \label{fig:binarna}
\end{figure}

Iz te slike je precej jasno, da so parametri na levi bolj primerni za hitro sintezo $b$ in $c$, če je
naš cilj seveda. Lahko pa sta to stranska produkta, ki ju ne želimo. V tem primeru je boljši desni primer,
kjer se $a^*$ ne zadržuje na $0$, a tam je treba biti previden, ker se $a^*$ poveča nato pa spet zmanjša, 
torej bi morali v tem primeru reakcijo zaustaviti, ko je $a^*$ na vrhuncu.\\

Če si pogledamo zdaj zahtevane rešitve, dobimo sliko \ref{fig:binarna-req}. Zdelo se mi je zanimivo,
da bi za pokazal še kako se bolj zvezno spreminja vsota $b + c$, zato sem jo tudi narisal. Tu bi bilo 
smiselno potem, da sta to produkta, ki ju želimo.\\

\begin{figure}[H]
    \centering
    \makebox[\textwidth][c]{%
    \includegraphics[width=1.2\textwidth]{../BinaryReactions/Images/exact-solution.png}
    }
    \caption{Točna rešitev.}
    \label{fig:binarna-req}
\end{figure}

Vidimo, da za majhne parametre $s$ hitrost limitira proti neki, recimo temu, kritični hitrosti. Tukaj 
sicer nisem uspel lepo narisati limitne hitrosti. Je pa to storjeno na kasnejšem primeru. Za velike parametre
pa hitrost limitira proti $0$. To je seveda pričakovano, saj parameter $s$ otežuje reakcijo iz $a^*$ v $b$ in $c$. \\

V približku ravnovesnega stanja pa sem dobil sliko \ref{fig:binarna-stac}. Tu se koncentracija zelo hitro
izravna, torej lahko z parametrom $s$ zelo dobro vplivamo na koncentracijo $b$ in $c$ v končnem stanju. To bi 
bilo koristno. če bi bila ta reakcija precursor za nek daljši proces.\\

\begin{figure}[H]
    \centering
    \makebox[\textwidth][c]{%
    \includegraphics[width=1.2\textwidth]{../BinaryReactions/Images/equilibrium-solution.png}
    }
    \caption{Rešitev v približku ravnovesnega stanja.}
    \label{fig:binarna-stac}
\end{figure}

Tu pa sem uspel dobiti limitno krivuljo, tako da sem vzel $s = 10^{100}$ (in še velikostne rede okoli) in ugotovil,
da se družina krivulj zbere v krivulji, ki kaže proti vsoti $0.5$. Zdaj pri drugi evalvaciji mi sicer da misliti, da sem
se mogoče pri normalizaciji rešitve zmotil za faktor $2$, ker bi morala vsota koncentracij biti $1$. Ampak recimo, 
da se razume, kaj je mišljeno.\\



\section{Komentarji in izboljšave}

\newpage
\bibliographystyle{unsrt}
\bibliography{sources}
\end{document}
