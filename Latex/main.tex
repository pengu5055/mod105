\documentclass[a4paper]{article}
\usepackage[utf8]{inputenc}
\usepackage[slovene]{babel}
\usepackage{graphicx}
\usepackage{hyperref}
\usepackage[nottoc]{tocbibind}
\usepackage{caption}
\usepackage{subcaption}
\usepackage{amsmath}
\usepackage{ dsfont }
\usepackage{siunitx}
\usepackage{multimedia}
\usepackage[table,xcdraw]{xcolor}
\setlength\parindent{0pt}

\newcommand{\ddd}{\mathrm{d}}
\newcommand\myworries[1]{\textcolor{red}{#1}}
\newcommand{\Dd}[3][{}]{\frac{\ddd^{#1} #2}{\ddd #3^{#1}}}

\begin{document}
\begin{titlepage}
    \begin{center}
        \includegraphics[]{logo.png}
        \vspace*{3cm}
        
        \Huge
        \textbf{Modeli kemijskih reakcij}
        
        \vspace{0.5cm}
        \large
        5. naloga pri Modelski Analizi 1

        \vspace{4.5cm}
        
        \textbf{Avtor:} Marko Urbanč (28232019)\ \\
        \textbf{Predavatelj:} prof. dr. Simon Širca\ \\
        \textbf{Asistent:} doc. dr. Miha Mihovilovič\ \\
        
        \vspace{2.8cm}
        
        \large
        8.11.2023
    \end{center}
\end{titlepage}
\tableofcontents
\newpage
\section{Uvod}
Čeprav smo fiziki in smo na študiju fizike, ne moremo zanikati, da je tudi kemija pomembna veda. A izkaže se, da 
ni vedno najbolj ugodno narediti prav vsak eksperiment v živo. Zato se je razvil računalniški pristop,
ki nam omogoča, da lahko simuliramo različne kemijske reakcije. Pravzaprav je pristop praktično analogen
prejšnji nalogi o populacijskih modelih. No vsaj ko naredimo ustrezne približke. V resnici, če bi hoteli brez kompromisa
modelirati kemijske procese, bi nas to drago stalo. To mislim dobesedno, ker če bi reševali t.i. \textbf{Chemical master
equation} bi dobili sistem ODE s toliko komponent, kolikor je možnih stanj sistema. To pa je, če želimo simulirati nek
makroskopski proces, zelo veliko. Zato se bomo v tej nalogi omejili na reševanje \textbf{Reaction rate} enačb, ki so
približek prej omenjenih enačb. V tej nalogi bomo spoznali modeliranje kemijskih procesov na podlagi treh primerov.\\

\subsection{Binarna reakcija}
Za prvi primer si poglejmo primer binarne reakcije. Kemijske procese lahko opišemo kot:

\begin{gather}
    A + A \overset{p}{\underset{q}{\rightleftharpoons}} A + A^*\>,\\
    A^* \overset{r}{\rightarrow} B + C\>.
\end{gather}

Rate enačbe v brezdimenzijski obliki za proces dobimo tako, da vse koncentracije normiramo
z začetno koncentracijo snovi $A$ $[A](0)$. S tem bomo lahko uvedli še nove parametre. Takole:

\begin{gather}
    \dot{a} = \frac{1}{2} kaa^* - a^2 + \frac{1}{2}a^2\>,\\
    \dot{a^*} = \frac{1}{2}a^2 - \frac{1}{2} kaa^* - ska^*\>,\\
    \dot{b} = \frac{1}{2} ska^*\>,\\
    \dot{c} = \frac{1}{2} ska^*\>.
\end{gather}

kjer je $k = \frac{p}{q}$ in $s = \frac{r}{q[A](0)}$. Odvod pa je zdaj po $d\tau = p[A](0)\>dt$. Lahko
predpostavimo, da v ravnovesnem stanju velja $\dot{a^*} = 0$, kar nam omogoči, da izrazimo $a^*$ in s tem 
dobimo nov set enačb:

\begin{gather}
    \dot{a} = \frac{a^3}{4k(s + a/2)} - a^2\>,\\
    \dot{a^*} = 0\>,\\
    \dot{b} = \frac{sa^2}{4(s + a/2)}\>,\\
    \dot{c} = \frac{sa^2}{4(s + a/2)}\>.
\end{gather}

\subsection{Sinteza vodikovega bromida}
Za drugi primer pogledamo primer sinteze vodikovega bromida. Sinteza je sestavljena iz več stopnj. Te lahko
zapišemo kot:

\begin{gather}
    \mathrm{Br}_2 \overset{p}{\underset{q}{\rightleftharpoons}} 2\mathrm{Br}\>,\\
    \mathrm{Br} + \mathrm{H}_2 \overset{r}{\underset{s}{\rightleftharpoons}} \mathrm{HBr} + \mathrm{H}\>,\\
    \mathrm{H} + \mathrm{Br}_2 \overset{t}{\rightarrow} \mathrm{HBr} + \mathrm{Br}\>.
\end{gather}

To lahko z sistemom ODE opišemo kot prej takole:

\begin{gather}
    \dot{u} = sxy - ruz\>,\\
    \dot{v} = qz^2 - pv - \frac{1}{2}tvy\>,\\
    \dot{x} = \frac{1}{2}tvy - \frac{1}{2}sxy + \frac{1}{2}ruz\>,\\
    \dot{y} = \frac{1}{2}ruz - \frac{1}{2}sxy + \frac{1}{2}tvy\>,\\
    \dot{z} = \frac{1}{2}tvy + pv - qz^2\>,
\end{gather}

kjer je $u = [\mathrm{H}_2]$, $v = [\mathrm{Br}_2]$, $x = [\mathrm{HBr}]$, $y = [\mathrm{H}]$ in $z = [\mathrm{Br}]$.
Tudi ta sistem lahko zapišemo v približku stacionarnega stanja, kjer velja $\dot{y} = 0$ in dobimo:

\begin{gather}
    \dot{u} = -\frac{1}{2}tvy\>,\\
    \dot{v} = qz^2 - pv - \frac{1}{2}tvy\>,\\
    \dot{x} = tvy\>,\\
    \dot{y} = 0\>,\\
    \dot{z} = \frac{1}{2}tvy + pv - qz^2\>.
\end{gather}

Na predavanjih je profesor omenil še možnost stacionarnega stanja za $\dot{z} = 0$, ampak meni ni uspelo
uspešno pognati reakcije v tem primeru.\\

\subsection{Kemijska ura}
Za konec pa še bolj zabaven primer, kemijska ura. Princip je, da imamo dve reakciji, ki se odvijata vzporedno.
Prva reakcija je hitra, druga pa počasna. Ko se prva reakcija konča, se začne druga in s tem se spremeni barva
raztopine. Pravzaprav imamo dve reakciji za vsako od barv, torej dve hitri in dve počasni. To lahko zapišemo kot:

\begin{gather}
    \mathrm{S}\mathrm{O}_8^{2-} + \mathrm{I}^- \overset{p_\mathrm{slow}}{\longrightarrow} \mathrm{I}\mathrm{S}_2\mathrm{O}_8^{3-}\>,\\
    2\mathrm{S}_2\mathrm{O}_3^{2-} + \mathrm{I}_2 \overset{p_\mathrm{fast}}{\longrightarrow} \mathrm{S}_4\mathrm{O}_6^{2-} + 2\mathrm{I}^-\>,\\
    \mathrm{S}_{2}\mathrm{O}_3^{2-} + \mathrm{I_2} \overset{q_\mathrm{slow}}{\longrightarrow} \mathrm{I}\mathrm{S}_2\mathrm{O}_3^{-} + \mathrm{I}^-\>,\\
    \mathrm{I}\mathrm{S}_2\mathrm{O}_3^{-} + \mathrm{S}_2\mathrm{O}_3^{2-} \overset{q_\mathrm{fast}}{\longrightarrow} \mathrm{S}_4\mathrm{O}_6^{2-} + \mathrm{I}^-\>.
\end{gather}

\section{Naloga}

\section{Opis reševanja}

\section{Rezultati}


\section{Komentarji in izboljšave}

\newpage
\bibliographystyle{unsrt}
\bibliography{sources}
\end{document}
